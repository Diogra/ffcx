\chapter{Interface specification}
\label{sec:interface}

% Write about all functions are const except a few and why
% Write about the order in which we present the interface


\section{Cell shapes}

%%  /// Valid cell shapes
%%  enum shape {interval, triangle, quadrilateral, tetrahedron, hexahedron};

\section{The class \texttt{ufc::mesh}}

The class \texttt{ufc::mesh} defines the data structure for a finite element mesh.

% Write: it's a view, not a data structure

% Write: it's a view, not a data structure

\subsection{The function \texttt{topological\_dimension()}}

This function takes no arguments and returns an unsigned integer
specifying the topological dimension of the mesh, that is the
topological dimension of the cells of the mesh. For the supported
cell shapes defined above, the topological dimensions are as follows:
\texttt{interval} has dimension 0, \texttt{triangle} and
\texttt{quadrilateral} have dimension 2, and tetrahedron and
hexahedron have dimension 3.

\subsection{The function \texttt{geometric\_dimension()}}

%%     /// Geometric dimension of the mesh
%%     unsigned int geometric_dimension;

\subsection{The array \texttt{num\_entities}}

%%     /// Array of the global number of entities of each topological dimension
%%     unsigned int* num_entities;

\section{The class \texttt{ufc::cell}}

The class \texttt{ufc::cell} defines the data structure for a cell in a mesh.

% Write: it's a view, not a data structure

%%   class cell
%%   {
%%   public:

%%     /// Constructor
%%     cell(): cell_shape(interval),
%%             topological_dimension(0), geometric_dimension(0),
%%             entity_indices(0), coordinates(0) {}

%%     /// Destructor
%%     virtual ~cell() {}

%%     /// Shape of the cell
%%     shape cell_shape;

%%     /// Topological dimension of the mesh
%%     unsigned int topological_dimension;

%%     /// Geometric dimension of the mesh
%%     unsigned int geometric_dimension;

%%     /// Array of global indices for the mesh entities of the cell
%%     unsigned int** entity_indices;

%%     /// Array of coordinates for the vertices of the cell
%%     double** coordinates;

%%   };

\section{The class \texttt{ufc::function}}

%%   /// This class defines the interface for a general tensor-valued function.

%%   class function
%%   {
%%   public:

%%     /// Constructor
%%     function() {}

%%     /// Destructor
%%     virtual ~function() {}

%%     /// Evaluate function at given point in cell
%%     virtual void evaluate(double* values,
%%                           const double* coordinates,
%%                           const cell& c) const = 0;

%%   };

\section{The class \texttt{ufc::finite\_element}}

%%   /// This class defines the interface for a finite element.

%%   class finite_element
%%   {
%%   public:

%%     /// Constructor
%%     finite_element() {}

%%     /// Destructor
%%     virtual ~finite_element() {}

%%     /// Return a string identifying the finite element
%%     virtual const char* signature() const = 0;

%%     /// Return the cell shape
%%     virtual shape cell_shape() const = 0;

%%     /// Return the dimension of the finite element function space
%%     virtual unsigned int space_dimension() const = 0;

%%     /// Return the rank of the value space
%%     virtual unsigned int value_rank() const = 0;

%%     /// Return the dimension of the value space for axis i
%%     virtual unsigned int value_dimension(unsigned int i) const = 0;

%%     /// Evaluate basis function i at given point in cell
%%     virtual void evaluate_basis(unsigned int i,
%%                                 double* values,
%%                                 const double* coordinates,
%%                                 const cell& c) const = 0;

%%     /// Evaluate order n derivatives of basis function i at given point in cell
%%     virtual void evaluate_basis_derivatives(unsigned int i,
%%                                             unsigned int n,
%%                                             double* values,
%%                                             const double* coordinates,
%%                                             const ufc::cell& c) const = 0;

%%     /// Evaluate linear functional for dof i on the function f
%%     virtual double evaluate_dof(unsigned int i,
%%                                 const function& f,
%%                                 const cell& c) const = 0;

%%     /// Interpolate vertex values from dof values
%%     virtual void interpolate_vertex_values(double* vertex_values,
%%                                            const double* dof_values,
%%                                            const cell& c) const = 0;

%%     /// Return the number of sub elements (for a mixed element)
%%     virtual unsigned int num_sub_elements() const = 0;

%%     /// Create a new finite element for sub element i (for a mixed element)
%%     virtual finite_element* create_sub_element(unsigned int i) const = 0;

%%   };

\section{The class \texttt{ufc::dof\_map}}

%%   /// This class defines the interface for a local-to-global mapping of
%%   /// degrees of freedom (dofs).

%%   class dof_map
%%   {
%%   public:

%%     /// Constructor
%%     dof_map() {}

%%     /// Destructor
%%     virtual ~dof_map() {}

%%     /// Return a string identifying the dof map
%%     virtual const char* signature() const = 0;

%%     /// Return true iff mesh entities of topological dimension d are needed
%%     virtual bool needs_mesh_entities(unsigned int d) const = 0;

%%     /// Initialize dof map for mesh (return true iff init_cell() is needed)
%%     virtual bool init_mesh(const mesh& mesh) = 0;

%%     /// Initialize dof map for given cell
%%     virtual void init_cell(const mesh& m,
%%                            const cell& c) = 0;

%%     /// Finish initialization of dof map for cells
%%     virtual void init_cell_finalize() = 0;

%%     /// Return the dimension of the global finite element function space
%%     virtual unsigned int global_dimension() const = 0;

%%     /// Return the dimension of the local finite element function space
%%     virtual unsigned int local_dimension() const = 0;

%%     /// Return the number of dofs on each cell facet
%%     virtual unsigned int num_facet_dofs() const = 0;

%%     /// Tabulate the local-to-global mapping of dofs on a cell
%%     virtual void tabulate_dofs(unsigned int* dofs,
%%                                const mesh& m,
%%                                const cell& c) const = 0;

%%     /// Tabulate the local-to-local mapping from facet dofs to cell dofs
%%     virtual void tabulate_facet_dofs(unsigned int* dofs,
%%                                      const mesh& m,
%%                                      const cell& c,
%%                                      unsigned int facet) const = 0;

%%     /// Tabulate the coordinates of all dofs on a cell
%%     virtual void tabulate_coordinates(double **coordinates,
%%                                       const cell& c) const = 0;

%%     /// Return the number of sub dof maps (for a mixed element)
%%     virtual unsigned int num_sub_dof_maps() const = 0;

%%     /// Create a new dof_map for sub dof map i (for a mixed element)
%%     virtual dof_map* create_sub_dof_map(unsigned int i) const = 0;

%%   };

\section{The class \texttt{ufc::cell\_integral}}

%%   /// This class defines the interface for the tabulation of the cell
%%   /// tensor corresponding to the local contribution to a form from
%%   /// the integral over a cell.

%%   class cell_integral
%%   {
%%   public:

%%     /// Constructor
%%     cell_integral() {}

%%     /// Destructor
%%     virtual ~cell_integral() {}

%%     /// Tabulate the tensor for the contribution from a local cell
%%     virtual void tabulate_tensor(double* A,
%%                                  const double * const * w,
%%                                  const cell& c) const = 0;

%%   };

\section{The class \texttt{ufc::exterior\_facet\_integral}}

%%   /// This class defines the interface for the tabulation of the
%%   /// exterior facet tensor corresponding to the local contribution to
%%   /// a form from the integral over an exterior facet.

%%   class exterior_facet_integral
%%   {
%%   public:

%%     /// Constructor
%%     exterior_facet_integral() {}

%%     /// Destructor
%%     virtual ~exterior_facet_integral() {}

%%     /// Tabulate the tensor for the contribution from a local exterior facet
%%     virtual void tabulate_tensor(double* A,
%%                                  const double * const * w,
%%                                  const cell& c,
%%                                  unsigned int facet) const = 0;

%%   };

\section{The class \texttt{ufc::interior\_facet\_integral}}

%%   /// This class defines the interface for the tabulation of the
%%   /// interior facet tensor corresponding to the local contribution to
%%   /// a form from the integral over an interior facet.

%%   class interior_facet_integral
%%   {
%%   public:

%%     /// Constructor
%%     interior_facet_integral() {}

%%     /// Destructor
%%     virtual ~interior_facet_integral() {}

%%     /// Tabulate the tensor for the contribution from a local interior facet
%%     virtual void tabulate_tensor(double* A,
%%                                  const double * const * w,
%%                                  const cell& c0,
%%                                  const cell& c1,
%%                                  unsigned int facet0,
%%                                  unsigned int facet1) const = 0;

%%   };

\section{The class \texttt{ufc::form}}

%%   /// This class defines the interface for the assembly of the global
%%   /// tensor corresponding to a form with r + n arguments, that is, a
%%   /// mapping
%%   ///
%%   ///     a : V1 x V2 x ... Vr x W1 x W2 x ... x Wn -> R
%%   ///
%%   /// with arguments v1, v2, ..., vr, w1, w2, ..., wn. The rank r
%%   /// global tensor A is defined by
%%   ///
%%   ///     A = a(V1, V2, ..., Vr, w1, w2, ..., wn),
%%   ///
%%   /// where each argument Vj represents the application to the
%%   /// sequence of basis functions of Vj and w1, w2, ..., wn are given
%%   /// fixed functions (coefficients).

%%   class form
%%   {
%%   public:

%%     /// Constructor
%%     form() {}

%%     /// Destructor
%%     virtual ~form() {}

%%     /// Return a string identifying the form
%%     virtual const char* signature() const = 0;

%%     /// Return the rank of the global tensor (r)
%%     virtual unsigned int rank() const = 0;

%%     /// Return the number of coefficients (n)
%%     virtual unsigned int num_coefficients() const = 0;

%%     /// Return the number of cell integrals
%%     virtual unsigned int num_cell_integrals() const = 0;

%%     /// Return the number of exterior facet integrals
%%     virtual unsigned int num_exterior_facet_integrals() const = 0;

%%     /// Return the number of interior facet integrals
%%     virtual unsigned int num_interior_facet_integrals() const = 0;

%%     /// Create a new finite element for argument function i
%%     virtual finite_element* create_finite_element(unsigned int i) const = 0;

%%     /// Create a new dof map for argument function i
%%     virtual dof_map* create_dof_map(unsigned int i) const = 0;

%%     /// Create a new cell integral on sub domain i
%%     virtual cell_integral* create_cell_integral(unsigned int i) const = 0;

%%     /// Create a new exterior facet integral on sub domain i
%%     virtual exterior_facet_integral* create_exterior_facet_integral(unsigned int i) const = 0;

%%     /// Create a new interior facet integral on sub domain i
%%     virtual interior_facet_integral* create_interior_facet_integral(unsigned int i) const = 0;

%%   };

%% }

%% #endif

% Need to write about the Python utils somewhere, should be in appendix