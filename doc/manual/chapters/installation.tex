\chapter{Installation}
\label{app:installation}
\index{installation}

The \ufc{} package consists of two parts, the main part being a single
header file called \texttt{ufc.h}. In addition, a set of Python
utilities to simplify the generation of \ufc{} code is provided.

Questions, bug reports and patches concerning the installation should
be directed to the \ufc{} mailing list at the address
\begin{code}
ufc-dev@fenics.org
\end{code}

\ufc{} must currently be installed directly from source, but Debian
(Ubuntu) packages will be available in the future, for \ufc{} and
other \fenics{} components.

\section{Dependencies and requirements}
\index{dependencies}

The installation system is based on distutils, and should work on any
system with a standard Python installation.

\subsection{Installing Python}

If Python is not installed on your system, it can be downloaded from
\begin{code}
http://www.python.org/
\end{code}
Follow the installation instructions for Python given on the Python
web page.  For Debian (Ubuntu) users, the package to install is named
\texttt{python}.

\section{Installing \ufc{}}

\ufc{} follows the standard installation procedure for Python
packages. Enter the source directory of \ufc{} and issue the following
command:
\begin{code}
# sudo python setup.py install
\end{code}
This will install the \ufc{} utility Python package in a subdirectory
called \texttt{ufc} in the default location for user-installed Python
packages (usually something like
\texttt{/usr/lib/python2.5/site-packages}). In addition, the header
file \texttt{ufc.h} will be installed.

To install \ufc{} under a different directory, for example if you
don't have root access to the system you're using, type
\begin{code}
# python setup.py install --prefix=~/local
\end{code}
This installs the \ufc{} header file in the directory
\texttt{\~{}/local/include} and the Python utilities under
\texttt{\~{}/local/lib/python}. If you use this option, make sure to
add \texttt{\~{}/local/lib/python} to the environment variable
\texttt{PYTHONPATH} and to add \texttt{\~{}/local/lib/pkgconfig} to
the \texttt{PKG\_CONFIG\_PATH} environment variable.

