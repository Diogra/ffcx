\chapter{Python interface}

\ffc{} provides a Python interface in the form of a standard
Python module. The following example demonstrates how to define and
compile the variational problem for Poisson's equation in a Python
script:

\begin{code}
 from ffc import *

 element = FiniteElement(``Lagrange'', ``triangle'', 1)

 v = BasisFunction(element)
 U = BasisFunction(element)
 f = Function(element)
  
 a = dot(grad(v), grad(U))*dx
 L = f*v*dx

 compile([a, L])
\end{code}

At the basic level, the only difference between the command-line
interface and the Python interface, is that the function
\texttt{compile} must be called when using the Python interface.

In addition to the function \texttt{compile}, the Python interface
provides the functions \texttt{build}, \texttt{write} and
\texttt{writeFiniteElement}. These functions are documented below.

Documentation can also be accessed from within Python. To read the
documentation for the function \texttt{compile}, run the following
commands in a Python shell:

\begin{code}
 from ffc import *
 help(compile)
\end{code}

\section{\texttt{compile(forms, ...)}}

This function takes as argument a form or list of forms
and compiles into low-level code for assembly. Calling this function
is equivalent to first calling \texttt{build} followed by
\texttt{write}.

\section{\texttt{build(forms, ...)}}

This function takes as argument a form or list of forms
and does preprocessing of the forms (including computation of the
reference tensor), but does not generate any code.

\section{\texttt{write(forms, ...)}}

This function takes a preprocessed form or list of forms and
generates code. Note that \texttt{build} must be called before
\texttt{write}:

\begin{code}
  forms = build([a, L])
  write(forms)
\end{code}

\section{\texttt{writeFiniteElement(element, ...)}}

This function generates code for a given \texttt{FiniteElement}.
Use this function if you just want to generate code for a finite
element (including mapping of nodes and nodal points):

\begin{code}
  element = FiniteElement(``Lagrange'', ``triangle'', 1)
  writeFiniteElement(element)
\end{code}
